\documentclass[a4paper, 11pt]{article}


\usepackage[utf8]{inputenc}
\usepackage[frenchb]{babel}
\usepackage[T1]{fontenc}
\usepackage{textcomp}
\usepackage{amsmath,amssymb}
\usepackage{lmodern}
\usepackage[a4paper]{geometry}
\usepackage{graphicx}
\usepackage{xcolor}
\usepackage{microtype}
\usepackage{listings}
\usepackage{hyperref}
\DeclareUnicodeCharacter{00A0}{ }

\renewcommand*{\familydefault}{\sfdefault}


\title{Aide à la Décision}


\author{Moreno Yassine \cr Mehdi Kitane \cr Amine ElRhazi \cr Abdelalim Tribak \cr Meryem Benchakroune \cr Karim Benhmida}

\begin{document}

\begin{LARGE}
\maketitle
\end{LARGE}

\tableofcontents
\newpage

\section{Introduction}
\subsection{Mise en situation}
\subsection{Concepts généraux et Contraintes}
\subsection*{Concepts généraux}
On considèrera dans tout ce qui suit : \\
- $X = \begin{pmatrix}
        a\\
        b\\
        c\\
        d\\
        e\\
        f\\
    \end{pmatrix}$ le vecteur définissant la quantité à fabriquer pour chaque produit (\textbf{variables de décision}).\\
- $Mchn = \left\{1, 2, 3, 4, 5, 6, 7\right\}$ l'ensemble des machines utilisées dans la fabrication des produits.\\
- $Mtr = \left\{MP1,MP2,MP3\right\}$ l'ensemble des matières premières utilisées lors de la confection des produits.\\
...
\subsection*{Jeu de contraintes}
... \\
\section{Programmation Linéaire Monocritère}
\subsection{Modélisation et Proposition de solution : Comptable}
L'objectif du comptable est de maximiser le bénéfice, ce dernier est obtenu par la différence entre le revenu global $G$ et des charges totales $C$  liées à l'achat des matières premières et du cout d'usinage sur les machines de l'entreprise.\\
Ainsi la fonction objectif du problème de comptabilité se modélise sous la forme suivante :
\begin{center}
$F_{compta} (X) = G(X) - C(X)$ \\
\end{center}
En considérant les fonctions suivantes : \\
- Le revenu global :\\
$G(X) = (20~27~26~30~45~40)\cdot X $ \\
$G(X) =  20\cdot a + 27\cdot b + 26\cdot c + 30\cdot d + 45\cdot e + 40\cdot f $ \\
- Les charges totales :\\
$C(X) = C_{mtrprem}(X) + C_{machines}(X) $ \\ 
Pour les matières premières :
$C_{mtrprem}(X) = \sum_{i\in{X}}{Cmtp_i\cdot i} $ \\

$Prix = \begin{pmatrix}
        3\\
        4\\
        2\\
    \end{pmatrix}$
$$
\begin{array}{r l}
    Cmtp_a = & (1~2~1)\cdot Prix = 13 \\
    Cmtp_b = & (2~2~0)\cdot Prix = 14 \\
    Cmtp_c = & (1~1~3)\cdot Prix = 15 \\
    Cmtp_d = & (5~2~2)\cdot Prix = 27 \\
    Cmtp_e = & (0~2~2)\cdot Prix = 12 \\
    Cmtp_f = & (2~1~0)\cdot Prix = 10 \\
\end{array}
$$
\\
D'où : $C_{mtrprem}(X) = 13\cdot a + 14\cdot b + 15\cdot c + 27\cdot d + 12\cdot e + 10\cdot f  $ \\
Pour les couts d'usinage :\\
$C_{machines}(X) = \sum_{i\in{X}}{Cmachines_i\cdot i} $ \\

$CoutMinute = \begin{pmatrix}
        0.03\\
        0.03\\
        0.017\\
        0.017\\
        0.03\\
        0.05\\
        0.05\\
    \end{pmatrix}$
$$
\begin{array}{r r l l}
    Cmachines_a = & (8~7~8~2~5~5~5)\cdot  & CoutMinute = & 1.33 \\
    Cmachines_b = & (15~1~1~10~0~5~3)\cdot   & CoutMinute = & 1.11 \\
    Cmachines_c = & (0~2~11~5~0~3~5)\cdot   & CoutMinute = & 0.73 \\
    Cmachines_d = & (5~15~0~4~7~12~8)\cdot   & CoutMinute = & 1.96 \\
    Cmachines_e = & (0~7~10~13~10~8~0)\cdot  & CoutMinute = & 1.35 \\
    Cmachines_f = & (10~12~25~7~25~0~7)\cdot & CoutMinute = & 2.45 \\
\end{array}
$$
D'où : $C_{machines}(X) = 1.33\cdot a + 1.11\cdot b + 0.73\cdot c + 1.96\cdot d + 1.35\cdot e + 2.45\cdot f  $ \\

Ainsi, nous pouvons déduire la fonction objectif -traduisant le bénéfice- à maximiser :
\begin{center}
$F_{compta} (X) = (5.67~12.38~12.27~1.03~31.65~27.55)\cdot X$ \\
\end{center}

De ce fait, nous avons effectué cette démarche en minimisant la fonction suivante tout en intégrant les contraintes spécifiés précédemment :
\begin{center}
$f(X) = -F_{compta} (X) = (-5.67~-12.38~-12.27~-1.03~-31.65~-27.55)\cdot X$ \\
\end{center}
$$
avec \left\{\begin{split}
	f(X)\ à\ minimiser\\
    A\cdot X \leq b\\
    0 \leq X
\end{split}\right.
$$

La solution proposée est la suivante :
$ X =\begin{pmatrix}
0\\
23.52\\
0\\
0\\
242.50\\
87.96
\end{pmatrix} $\\

Le bénéfice maximal envisageable est de : \textbf{10390 €}


\subsection{Modélisation et Proposition de solution : Resp.Atelier}
L'objectif du responsable d'atelier est de maximiser la fabrication des produits. Autrement dit, maximiser la quantité totale des produits. 
Nous modéliserons la fonction objectif sous la forme suivante :
\begin{center}
$F_{respAtelier} (X) =(1~1~1~1~1~1)\cdot X$ \\
\end{center}

Ainsi, le problème est réduit à une fonction $f$ à minimiser tout en tenant compte des contraintes globales précédentes :
$f(X) = F_{respAtelier} (X) =(-1~-1~-1~-1~-1~-1)\cdot X$ \\
$$
avec \left\{\begin{split}
	f(X)\ à\ minimiser\\
    A\cdot X \leq b\\
    0 \leq X
\end{split}\right.
$$

La solution proposée est la suivante :
$ X =\begin{pmatrix}
0\\
56.73\\
38.69\\
0\\
184.46\\
98.92
\end{pmatrix} $\\

La production maximale envisageable est de : \textbf{378.8 Produits}\\

\subsection{Modélisation et Proposition de solution : Resp.Stock}
La problématique du responsable des stocks est double, puisqu'il s'agit de minimiser le stock global de l'entreprise tout en préservant un niveau minimal d'activité au sein de l'entreprise.\\
Nous nous attaquerons dans un premier temps à la fonction objectif , obtenu par la somme des quantités des produits et des quantités de matières premières pour chaque produit :
\begin{center}
$F_{respStock} (X) =(1~1~1~1~1~1)+(4~4~5~9~4~3)\cdot X$ \\
$F_{respStock} (X) =(5~5~6~10~5~4)\cdot X$\\
\end{center}
Néanmoins, cette fonction -à minimiser- admet le vecteur nul comme solution minimale, réduisant à néant la productivité de l'entreprise.\\
De ce fait, afin de contourner le problème, nous ajouterons 2 contraintes supplémentaires concernant la production totale de l'entreprise pour traduire l'activité.\\
- La première contrainte, qui concerne la production maximale, se traduit par l'inégalité suivante :
\textbf{a+b+c+d+e+f < 378.8}\\
- La deuxième contrainte concerne un seuil minimal (pourcentage de la production maximale, à définir) d'activité :
\textbf{ 378.8*i < a+b+c+d+e+f avec 0<i<1}\\
	 
\subsection{Modélisation et Proposition de solution : Resp.Commercial}
\subsection{Modélisation et Proposition de solution : Resp.Personnel}

\section{Programmation Linéaire Multicritère}

\section{Partie3}
test

\section*{Annexe}
test
\end{document}