\documentclass[a4paper, 11pt]{article}


\usepackage[utf8]{inputenc}
\usepackage[frenchb]{babel}
\usepackage[T1]{fontenc}
\usepackage{textcomp}
\usepackage{amsmath,amssymb}
\usepackage{lmodern}
\usepackage[a4paper]{geometry}
\usepackage{graphicx}
\usepackage{xcolor}
\usepackage{microtype}
\usepackage{listings}
\usepackage{hyperref}
\DeclareUnicodeCharacter{00A0}{ }

\renewcommand*{\familydefault}{\sfdefault}


\title{Rapport D'Analyse}


\author{Yassine Moreno \cr Mehdi Kitane \cr Amine ElRhazi \cr Abdelalim Tribak \cr Meryem Benchakroune \cr Karim Benhmida}

\begin{document}

\begin{LARGE}
\maketitle
\end{LARGE}

\tableofcontents
\newpage

\section{Introduction}
\subsection{Mise en situation}
\subsection{Concepts généraux et Contraintes}
\subsection*{Concepts généraux}
On considèrera dans tout ce qui suit : \\
- $X = \begin{pmatrix}
        a\\
        b\\
        c\\
        d\\
        e\\
        f\\
    \end{pmatrix}$ le vecteur définissant la quantité hebdomadaire à fabriquer pour chaque produit (\textbf{variables de décision}).\\
- $Mchn = \left\{1, 2, 3, 4, 5, 6, 7\right\}$ l'ensemble des machines utilisées dans la fabrication des produits.\\
- $Mtr = \left\{MP1,MP2,MP3\right\}$ l'ensemble des matières premières utilisées lors de la confection des produits.\\
...
\subsection*{Jeu de contraintes}
Nous détaillerons dans cette section les contraintes imposées par les différents constituants de l'entreprise FaBrique. \\
Tout d'abord, les contraintes de domaine stipulent que toutes les quantités hebdomadaires à fabriquer par l'entreprise doivent être positifs ou nuls, ce qui peut se traduire par l'inégalité suivante :
\begin{center}
$$0 \leq X \Leftrightarrow 0 \leq a ,0 \leq b ,0 \leq c ,0 \leq d ,0 \leq e ,0 \leq f $$
\end{center}
Ensuite, les contraintes imposées par la quantité maximale de matières premières admissible : \textbf{350 unités pour MP1, 620 unités pour MP2 et 485 unités pour MP3}. Ces contraintes se traduiront par les inégalités suivantes :
$$
\left\{\begin{split}
	a+2b+c+5d+2f \leq 350 \\
    2a+2b+c+2d+2e+f  \leq 620 \\
    a+3c+2d+2e \leq 485 \\
\end{split}\right. \Leftrightarrow \begin{pmatrix}
        1&2&1&5&0&2 \\
        2&2&1&2&2&1\\
        1&0&3&2&2&0\\
    \end{pmatrix} \cdot X \leq \begin{pmatrix}
        350 \\
        620\\
        485\\
    \end{pmatrix}
$$
Enfin, les contraintes traduisant le volume horaire \textbf{hebdomadaire} maximal du fonctionnement des machines, ne dépassant pas 16 heures par jour, 5 jours par semaine -un total de \textbf{4800 minutes} hebdomadaires-.
$$
\left\{\begin{split}
	8a+15b+5d+10f \leq 4800 \\
    7a+b+2c+15d+7e+12f \leq 4800 \\
    8a+b+11c+10e+25f \leq 4800 \\
    2a+10b+5c+4d+13e+7f \leq 4800 \\
    5a+7d+10e+25f \leq 4800 \\
	5a+3b+5c+8d+7f \leq 4800\\
	5a+5b+3c+12d+8e \leq 4800\\
\end{split}\right. \Leftrightarrow \begin{pmatrix}
        8&15&0&5&0&10 \\
        7&1&2&15&7&12\\
        8&1&11&0&10&25\\
        2&10&5&4&13&7\\
        5&0&0&7&10&25\\
        5&3&5&8&0&7\\
        5&5&3&12&8&0
    \end{pmatrix} \cdot X \leq \begin{pmatrix}
        4800\\
        4800\\
        4800\\
        4800\\
        4800\\
        4800\\
        4800
    \end{pmatrix}
$$ 

\section{Programmation Linéaire Monocritère}
\subsection{Modélisation et Proposition de solution : Comptable}
L'objectif du comptable est de maximiser le bénéfice, ce dernier est obtenu par la différence entre le revenu global $G$ et des charges totales $C$  liées à l'achat des matières premières et du cout d'usinage sur les machines de l'entreprise.\\
Ainsi la fonction objectif du problème de comptabilité se modélise sous la forme suivante :
\begin{center}
$F_{compta} (X) = G(X) - C(X)$ \\
\end{center}
En considérant les fonctions suivantes : \\
- Le revenu global :\\
$G(X) = (20~27~26~30~45~40)\cdot X $ \\
$G(X) =  20\cdot a + 27\cdot b + 26\cdot c + 30\cdot d + 45\cdot e + 40\cdot f $ \\
- Les charges totales :\\
$C(X) = C_{mtrprem}(X) + C_{machines}(X) $ \\ 
Pour les matières premières :
$C_{mtrprem}(X) = \sum_{i\in{X}}{Cmtp_i\cdot i} $ \\
$Prix = \begin{pmatrix}
        3\\
        4\\
        2\\
    \end{pmatrix}$
$$
\begin{array}{r l}
    Cmtp_a = & (1~2~1)\cdot Prix = 13 \\
    Cmtp_b = & (2~2~0)\cdot Prix = 14 \\
    Cmtp_c = & (1~1~3)\cdot Prix = 15 \\
    Cmtp_d = & (5~2~2)\cdot Prix = 27 \\
    Cmtp_e = & (0~2~2)\cdot Prix = 12 \\
    Cmtp_f = & (2~1~0)\cdot Prix = 10 \\
\end{array}
$$
\\
D'où : $C_{mtrprem}(X) = 13\cdot a + 14\cdot b + 15\cdot c + 27\cdot d + 12\cdot e + 10\cdot f  $ \\
Pour les couts d'usinage :\\
$C_{machines}(X) = \sum_{i\in{X}}{Cmachines_i\cdot i} $ \\

$CoutMinute = \begin{pmatrix}
        0.03\\
        0.03\\
        0.017\\
        0.017\\
        0.03\\
        0.05\\
        0.05\\
    \end{pmatrix}$
$$
\begin{array}{r r l l}
    Cmachines_a = & (8~7~8~2~5~5~5)\cdot  & CoutMinute = & 1.33 \\
    Cmachines_b = & (15~1~1~10~0~5~3)\cdot   & CoutMinute = & 1.11 \\
    Cmachines_c = & (0~2~11~5~0~3~5)\cdot   & CoutMinute = & 0.73 \\
    Cmachines_d = & (5~15~0~4~7~12~8)\cdot   & CoutMinute = & 1.96 \\
    Cmachines_e = & (0~7~10~13~10~8~0)\cdot  & CoutMinute = & 1.35 \\
    Cmachines_f = & (10~12~25~7~25~0~7)\cdot & CoutMinute = & 2.45 \\
\end{array}
$$
D'où : $C_{machines}(X) = 1.33\cdot a + 1.11\cdot b + 0.73\cdot c + 1.96\cdot d + 1.35\cdot e + 2.45\cdot f  $ \\
Ainsi, nous pouvons déduire la fonction objectif -traduisant le bénéfice- à maximiser :
\begin{center}
$F_{compta} (X) = (5.67~12.38~12.27~1.03~31.65~27.55)\cdot X$ \\
\end{center}

De ce fait, nous avons effectué cette démarche en minimisant la fonction suivante tout en intégrant les contraintes spécifiés précédemment :
\begin{center}
$f(X) = -F_{compta} (X) = (-5.67~-11.88~-12.27~-1.03~-31.65~-27.55)\cdot X$ \\
\end{center}
$$
avec \left\{\begin{split}
	f(X)\ à\ minimiser\\
    A\cdot X \leq b\\
    0 \leq X
\end{split}\right.
$$

La solution proposée est la suivante :
$ X =\begin{pmatrix}
0\\
20.41\\
0\\
0\\
242.50\\
94.18
\end{pmatrix} $\\
Le bénéfice maximal envisageable est de : \textbf{10390 €}
\subsection{Modélisation et Proposition de solution : Resp.Atelier}
L'objectif du responsable d'atelier est de maximiser la fabrication des produits. Autrement dit, maximiser la quantité totale des produits. 
Nous modéliserons la fonction objectif sous la forme suivante :
\begin{center}
$F_{respAtelier} (X) =(1~1~1~1~1~1)\cdot X$ \\
\end{center}
Ainsi, le problème est réduit à une fonction $f$ à minimiser tout en tenant compte des contraintes globales précédentes :
$f(X) = -F_{respAtelier} (X) =(-1~-1~-1~-1~-1~-1)\cdot X$ \\
$$
avec \left\{\begin{split}
	f(X)\ à\ minimiser\\
    A\cdot X \leq b\\
    0 \leq X
\end{split}\right.
$$

La solution proposée est la suivante :
$ X =\begin{pmatrix}
0\\
56.73\\
38.69\\
0\\
184.46\\
98.92
\end{pmatrix} $\\
La production maximale envisageable est de : \textbf{378.8 Produits}
\subsection{Modélisation et Proposition de solution : Resp.Stock}
La problématique du responsable des stocks est double, puisqu'il s'agit de minimiser le stock global de l'entreprise tout en préservant un niveau minimal d'activité au sein de l'entreprise.\\
Nous nous attaquerons dans un premier temps à la fonction objectif , obtenu par la somme des quantités des produits et des quantités de matières premières pour chaque produit :
\begin{center}
$F_{respStock} (X) =(1~1~1~1~1~1)+(4~4~5~9~4~3)\cdot X$ \\
$F_{respStock} (X) =(5~5~6~10~5~4)\cdot X$\\
\end{center}
Néanmoins, cette fonction -à minimiser- admet le vecteur nul comme solution minimale, réduisant à néant la productivité de l'entreprise.\\
De ce fait, afin de contourner le problème, nous ajouterons 2 contraintes supplémentaires concernant la production totale de l'entreprise pour traduire l'activité.\\
- La première contrainte, qui concerne la production maximale, se traduit par l'inégalité suivante :
\textbf{a+b+c+d+e+f < 378.8}\\
- La deuxième contrainte concerne un seuil minimal (pourcentage de la production maximale, à définir) d'activité :
\textbf{ 378.8*i < a+b+c+d+e+f avec 0<i<1}\\
\\
En intégrant les nouvelles contraintes au jeu établi précédemment par le biais de \textbf{$A_{new}$} et de \textbf{$b_{new}$} , nous avons pu modéliser l'évolution du stock global $F_{respStock} (X)$ en fonction du pourcentage $\%prodMax$ de la production maximale imposé en seuil minimal d'activité (cf. Figure 1).\\
\begin{figure}[position]
    \begin{center}
        \includegraphics[scale=0.35]{Stock}
        \caption{
            \label{fig} Évolution du stock en fonction de \%prodMax
        }
    \end{center}
\end{figure}
Ainsi, nous constatons un point d'inflexion établi à \textbf{47\%} de la production maximale. Au delà, on constate une augmentation très rapide du stock qui ne satisfait nullement l'objectif fixé.
C'est à partir de ce point particulier que nous construirons notre solution au problème suivant :
$f(X) = F_{respStock} (X) =(5~5~6~10~5~4)\cdot X$ \\
$$
avec \left\{\begin{split}
	f(X)\ à\ minimiser\\
    A_{new}\cdot X \leq b_{new}\\
    0 \leq X
\end{split}\right.
$$
La solution optimale pour ce problème est la suivante :
$ X =\begin{pmatrix}
38.35\\
25.47\\
0\\
0\\
108.87\\
130.36
\end{pmatrix} $\\
Le stock minimal obtenu par ces données est de : \textbf{1384 unités}

\subsection{Modélisation et Proposition de solution : Resp.Commercial}
Le responsable commercial a pour objectif la réalisation de l'équilibre entre la première famille de produits (A,B et C) et la seconde (D,E et F). Cette contrainte d'égalité se modélise par l'équation suivante :
\begin{center}
$ a+ b+ c = d+ e+ f\Leftrightarrow(1~1~1~-1~-1~-1)\cdot X = 0 \Leftrightarrow A_{eq}\cdot X = b_{eq} $
\end{center}
Le fonction objectif du responsable commercial demeure identique à celle du responsable d'atelier, maximiser la production de l'entreprise -pour assurer l'activité de l'entreprise- tout en tenant compte de la nouvelle contrainte d'équilibre.
\begin{center}
$F_{respComm} (X) =(1~1~1~1~1~1)\cdot X$ \\
\end{center}
Ainsi, le problème est mathématiquement réduit à une fonction $f$ à minimiser tout en tenant compte des contraintes globales précédentes en plus de la nouvelle contrainte d'équilibre :
$f(X) = -F_{respComm} (X) =(-1~-1~-1~-1~-1~-1)\cdot X$ \\
$$
avec \left\{\begin{split}
	f(X)\ à\ minimiser\\
    A\cdot X \leq b\\
    A_{eq}\cdot X = b_{eq}\\
    0 \leq X
\end{split}\right.
$$
La solution optimale pour ce problème est la suivante :
$ X =\begin{pmatrix}
142.11\\
0\\
44.42\\
0\\
104.80\\
81.73
\end{pmatrix} $\\
La production de la première famille s'élève à \textbf{186.53 unités}\\
La production de la deuxième famille s'élève à \textbf{186.53 unités}\\
L'écart théorique constaté entre ces deux familles est de : \textbf{0 unités}
\subsection{Modélisation et Proposition de solution : Resp.Personnel}
La problématique du responsable du personnel concerne la réduction de l'usage des machines 3 et 5 tout en maintenant une activité correcte de l'entreprise.
La fonction objectif modélisant le problème se retranscrit sous la forme suivante :
\begin{center}
$F_{respPers} (X) =F_{TmpsMachine3} (X) + F_{TmpsMachine5} (X)$\\
$F_{TmpsMachine3} (X) = (8~1~11~0~10~25)\cdot X$\\
$F_{TmpsMachine5} (X) = (5~0~0~7~10~25)\cdot X$\\
$F_{respPers} (X) =(13~1~11~7~20~50)\cdot X$
\end{center}

Cependant, le minimum envisageable pour cette fonction est le vecteur nul, solution absurde pour assurer l'activité de l'entreprise. Ainsi, de manière identique au problème du responsable des stocks, nous ajouterons 2 contraintes supplémentaires liées à l'activité maximale de l'entreprise.
- La première contrainte, qui concerne la production maximale, se traduit par l'inégalité suivante :
\textbf{a+b+c+d+e+f < 378.8}\\
- La deuxième contrainte concerne un seuil minimal (pourcentage de la production maximale, à définir) d'activité :
\textbf{ 378.8*i < a+b+c+d+e+f avec 0<i<1}\\
A partir du jeu de contraintes modifié interprété par \textbf{$A_{new}$} et de \textbf{$b_{new}$}, nous pouvons établir une relation traduisant l'impact du seuil minimal sur les temps d'usinage des machines 3 et 5 ainsi que sur le temps d'utilisation global (cf. Figure 2)\\
%\begin{figure}[position]
 %   \begin{center}
  %      \includegraphics[scale=0.35]{Stock}
   %     \caption{
    %        \label{fig} Évolution du stock en fonction de \%prodMax
    %    }
    %\end{center}
%\end{figure}
A partir de l'analyse des graphiques produits, nous déduisons que la solution optimale pour cette problématique s'établit à \textbf{82\%} de la production maximale, assurant un compromis idéal entre l'utilisation maximale des deux machines et l'utilisation indépendante de chacune d'entre elles.
Cette condition supplémentaire nous permettra de construire notre solution au problème suivant :
$f(X) = F_{respPers} (X) =(13~1~11~7~20~50)\cdot X$ \\
$$
avec \left\{\begin{split}
	f(X)\ à\ minimiser\\
    A_{new}\cdot X \leq b_{new}\\
    0 \leq X
\end{split}\right.
$$
La solution optimale pour ce problème est la suivante :
$ X =\begin{pmatrix}
0\\
174.38\\
1.23\\
0\\
135\\
0
\end{pmatrix} $\\
Le temps total minimal obtenu est de : \textbf{2887.9 mn}\\
Le temps minimal de la machine 3 obtenu est de : \textbf{1537.9 mn}\\
Le temps minimal de la machine 5 obtenu est de : \textbf{1350 mn}

\section{Programmation Linéaire Multicritère}
\subsection{Introduction}
Nous considérons à present le point de vue du directeur d’entreprise. Son objectif est de contenter tous les responsables en satisfaisant au mieux possible les critères de chacun.

\subsection{Matrice de Gain}
Nous commençons par representer la matrice de Gain.\\
La matrice de Gain correspond aux valeurs obtenues pour les objectifs de chaque responsable en fonction du cas idéal d’un responsable. Elle est générée à partir des résultats de la partie 1, c’est à dire aux choix d’une programmation mono-critère.
En récupérant les fonctions objectifs de chaque responsable et en récupérant les solutions optimales de chaque responsable, nous pouvons donc calculer la matrice de gain.

$ Gain=\begin{pmatrix}
10512 &357 &1691 &-316 &9579\\
9712 &379 &1834 &-188 &9118\\
7557 &303 &1385 &-175 &9219\\
6920 &373 &1828 &0 &8519\\
6359 &311 &1554 &41 &2888\\
\end{pmatrix} $\\


La première colonne correspond au benefice obtenu en tenant compte de la solution optimale de chaque responsable.
La deuxième correspond quand à elle au nombre de produit à fabriquer.
La troisième quand à elle correspond au nombre de produit dans le stock tandis que la quatrième correspond à la difference de quantité de produit fabriqué entre les deux familles.
Finalement, la 5ème correspond aux temps d’utilisation des machines 3 et 5.

La diagonal correspond au point de Mire, c’est à dire, le résultat optimal que voudrais atteindre le responsable de l’entreprise. Nous devons donc essayer de nous y approcher le plus possible.

$ PM = \begin{pmatrix}
10512\\
378.8\\
1385\\
0\\
2887.9\\
\end{pmatrix} $

\subsection{Application de la Programation Lineaire Multicritere}
Nous allons à present considerer le bénéfice de l’entreprise comme variable à modifier et placer les autres variable en contrainte. Nous pouvons alors jouer sur les différentes valeurs pour essayer de maximiser les différentes valeurs.

La matrice de contrainte A et le vecteur B deviennent : 

$ A = \begin{pmatrix}
1 &2 &1 &5 &0 &2\\
2 &2 &1 &2 &2 &1\\
1 &0 &3 &2 &2 &0\\
8 &15 &0 &5 &0 &10\\
7 &1 &2 &15 &7 &12\\
8 &1 &11 &0 &10 &25\\
2 &10 &5 &4 &13 &7\\
5 &0 &0 &7 &10 &25\\
5 &3 &5 &8 &0 &7\\
5 &5 &3 &12 &8 &0\\
-1 &-1 &-1 &-1 &-1 &-1\\      %resp Atelier
-5 &-5 &-6 &-10 &-5 &-4\\     %resp Stock
-1 &-1 &-1 &1 &1 &1\\         %resp Comm
-13 &-1 &-11 &-7 &-20 &-50\\  %resp Perso
 \end{pmatrix}  B = \begin{pmatrix}
350\\
620\\
485\\
4800\\
4800\\
4800\\
4800\\
4800\\
4800\\
4800\\
-378.8\\
-1385\\
-0\\
-2887.9 \\
 \end{pmatrix} $
 
Nous allons donc jouer sur les valeurs de B pour essayer de maximiser les chiffres.
Nous remarquons que les valeurs sont corrélées, plus nous maximisons un des différents objectifs plus les autres diminuent. Il faut alors essayer de trouver un compromis qui satisfera au mieux les différents objectifs.


 
Bien entendu, comme nous sommes en face d’une entreprise, le bénéfice et le profit sont parmis les variables à plus prendre en compte.  
Nous avons opté alors pour les valeurs suivantes. 
$ B = \begin{pmatrix}
350\\
620\\
485\\
4800\\
4800\\
4800\\
4800\\
4800\\
4800\\
4800\\
-339\\
-1425\\
40\\
-3075 \\
 \end{pmatrix} $
 
Nous optenons alors un nombre de produits fabriqué de l'ordre de : \\
0 produits A fabriqués \\
102.9687 produits B fabriqués \\
57.1875 produits C fabriqués \\
0 produits D fabriqués \\
156.7187 produits E fabriqués \\
43.4375 produits F fabriqués \\
 
 Ceci nous donne ainsi :\\
8081 de bénéfice\\
360 de produits fabriqués\\
1815 de produits dans le stock\\
40 de difference entre les produits de la famille A et B\\ 
6038 de temps d'utilisation des machines 3 et 5\\

Ceci permet de satisfaire : \\
77\% des objectifs du Comptable\\
95\% des objectifs du responsable Atelier\\
77\% des objectifs du responsable Stock\\
80\% des objectifs du reponsable commercial\\
et finalement 48\% des objectifs du responsable personnel\\
(Cf Figure 2)

\begin{figure}[position]
    \begin{center}
        \includegraphics[scale=0.55]{../Partie2/Partie2.png}
        \caption{
            \label{fig} Niveau de satisfaction en fonction des objectifs de chaque responsable
        }
    \end{center}
\end{figure}


Nous trouvons cette solution satisfaisante car nous rognons seulement sur les objectifs du responsable personnel pour obtenir environ 80\% des objectifs des autres responsables. 


 



\section{Partie3}
test

\section*{Annexe}
test
\end{document}